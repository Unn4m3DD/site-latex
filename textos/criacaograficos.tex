O ambiente \LaTeX \,oferece a possibilidade de criar gráficos na própria linguagem sem necessidade de incluir imagens externas de outros sites por exemplo Geogebra ou Desmos.\\

Para iniciar um ambiente de gráfico ortonormado de coordenadas entre x e y insere-se o comando "\textbackslash begin\{picture\}(x,y) ... \textbackslash end\{picture\}". Para que o gráfico seja visível em unidades predefinidas pode utilizar-se o comando "\textbackslash setlength{\textbackslash unitlength\}\{tamanho\}" com por exemplo 5cm. Por fim o comando para desenhar um linha é definido pelo vetor origem, o vetor direção e o tamanho da linha. O comando para instanciar a linha é "\textbackslash  put(VetorOrigem)\{\textbackslash line (VetorDireção)\{tamanhoDaLinha\}\}". Exemplificando a composição dos comandos descritos:\\\\\\

\begin{center}

\setlength{\unitlength}{5cm}
\begin{picture}(1,1)
\put(0,0){\line(0,1){1}}
\put(1,0){\line(-1,0){1}}
\put(0,1){\line(1,0){1}}
\put(1,1){\line(0,-1){1}}
\end{picture}

\end{center}

Gerado com o código:
\begin{verbatim}
\begin{center}

\setlength{\unitlength}{5cm}
\begin{picture}(1,1)
\put(0,0){\line(0,1){1}}
\put(1,0){\line(-1,0){1}}
\put(0,1){\line(1,0){1}}
\put(1,1){\line(0,-1){1}}
\end{picture}

\end{center}
\end{verbatim}


Esta é a forma mais simples de criação de gráficos em \LaTeX , e tem as suas variações por exemplo com o comando "\textbackslash vector" em vez do "\textbackslash line", que vai criar uma variação com setas em vez de linhas:

\begin{center}

\setlength{\unitlength}{5cm}
\begin{picture}(1,1)
\put(0,0){\vector(0,1){1}}
\put(1,0){\vector(-1,0){1}}
\put(0,1){\vector(1,0){1}}
\put(1,1){\vector(0,-1){1}}
\end{picture}

\end{center}

Gerado com o código:
\begin{verbatim}
\begin{center}

\setlength{\unitlength}{5cm}
\begin{picture}(1,1)
\put(0,0){\vector(0,1){1}}
\put(1,0){\vector(-1,0){1}}
\put(0,1){\vector(1,0){1}}
\put(1,1){\vector(0,-1){1}}
\end{picture}

\end{center}
\end{verbatim}

Podem criar-se circunferências com o comando "\textbackslash circle\{raio\}" e círculos com o comando "\textbackslash circle*\{raio\}":

\begin{center}
\setlength{\unitlength}{5mm}
\begin{picture}(10,10)
%desenhar quadrado
\put(0,0){\line(0,10){10}}
\put(10,0){\line(-10,0){10}}
\put(0,10){\line(10,0){10}}
\put(10,10){\line(0,-10){10}}
%desenhar Círculos
\put(5,5){\circle{10}}
\put(5,5){\circle*{1}}

\end{picture}
\end{center}

Gerado com o código:
\begin{verbatim}
\begin{center}
\setlength{\unitlength}{5mm}
\begin{picture}(10,10)
%desenhar quadrado
\put(0,0){\line(0,10){10}}
\put(10,0){\line(-10,0){10}}
\put(0,10){\line(10,0){10}}
\put(10,10){\line(0,-10){10}}
%desenhar Círculos
\put(5,5){\circle{10}}
\put(5,5){\circle*{1}}

\end{picture}
\end{center}
\end{verbatim}


É possível também escrever textos nos gráficos gerados, assim, por exemplo um triângulo pode ter vértices nomeados. O comando para tal efeito é "\textbackslash put(Origem)\{\$texto\$\}". Exemplificando:

\begin{center}
\setlength{\unitlength}{5mm}
\begin{picture}(10,10)
%desenhar quadrado
\put(0,0){\line(0,10){10}}
\put(10,0){\line(-10,0){10}}
\put(0,10){\line(10,0){10}}
\put(10,10){\line(0,-10){10}}
%desenhar triângulo
\put(3,3){\line(1,0){3}}
\put(6,3){\line(0,1){4}}
\put(6,7){\line(-3,-4){3}}
%Letras Vértices
\put(2.5,2.5){$A$}
\put(6,2.5){$B$}
\put(6,7){$C$}
\end{picture}

\end{center}



Gerado com o código:
\begin{verbatim}
\begin{center}
\setlength{\unitlength}{5mm}
\begin{picture}(10,10)
%desenhar quadrado
\put(0,0){\line(0,10){10}}
\put(10,0){\line(-10,0){10}}
\put(0,10){\line(10,0){10}}
\put(10,10){\line(0,-10){10}}
%desenhar triângulo
\put(3,3){\line(1,0){3}}
\put(6,3){\line(0,1){4}}
\put(6,7){\line(-3,-4){3}}
%Letras Vértices
\put(2.5,2.5){$A$}
\put(6,2.5){$B$}
\put(6,7){$C$}
\end{picture}

\end{center}
\end{verbatim}

Para além destas existe também a funcionalidade de escrever arrays de objetos como linhas ou círculos com o comando "\textbackslash multiput(x,y)($\Delta$ x,$\Delta$ y)\\{n\}\{objeto\}. Por exemplo:

\begin{center}

\setlength{\unitlength}{5mm}
\begin{picture}(10,10)

%desenhar quadrado
\put(0,0){\line(0,10){10}}
\put(10,0){\line(-10,0){10}}
\put(0,10){\line(10,0){10}}
\put(10,10){\line(0,-10){10}}
%Desenhar Array de Vetores
\multiput(1,5)(2,0){5}{\vector(0,-1){5}}
%Desenhar Array de Círculos
\multiput(1,5)(2,0){5}{\circle{2}}
\end{picture}
\end{center}

Gerado com o código:
\begin{verbatim}
\begin{center}

\setlength{\unitlength}{5mm}
\begin{picture}(10,10)

%desenhar quadrado
\put(0,0){\line(0,10){10}}
\put(10,0){\line(-10,0){10}}
\put(0,10){\line(10,0){10}}
\put(10,10){\line(0,-10){10}}
%Desenhar Array de Vetores
\multiput(1,5)(2,0){5}{\vector(0,-1){5}}
%Desenhar Array de Círculos
\multiput(1,5)(2,0){5}{\circle{2}}
\end{picture}
\end{center}
\end{verbatim}

