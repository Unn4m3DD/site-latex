O \LaTeX define um conjunto básico de macros para edição de textos. Caso o utilizador queria usar alguma função mais complexa, o \LaTeX permite que ele inclua arquivos com novos macros. Esses arquivos são chamados de pacotes. Existem pacotes para escrever a cor , para incluir figuras , etc ...
O utilizador pode ainda até criar o seu próprio pacote. 

\begin{verbatim}
Para incluir um pacote basta usar :
\usepackage[opção]{nome do pacote}
\end{verbatim}

Alguns pacotes podem ser incluídos usando opções diferentes . A opção deve ser inserida entre parêntesis retos antes do nome do pacote.Neste exemplo :

\begin{verbatim}
\usepackage[portuguese]{babel}
\end{verbatim}

o pacote babel define macros para edição de textos em diversas línguas . Como queríamos escrever em Português pusemos a opção portuguese. As distribuições do \LaTeX costumam vir com um conjunto amplo de pacotes. Outros pacotes podem ser instalados da mesma forma.