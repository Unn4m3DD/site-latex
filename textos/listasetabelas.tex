O \LaTeX oferece a possibilidade de criar listas e tabelas para facilitar a leitura e compreensão do texto.

\subsection{Listas}
Existe dois tipos de listas em \LaTeX, listas ordenadas e não ordenadas. Para criação de listas não ordenadas utiliza-se o comando "\textbackslash begin\{itemize\} ... \textbackslash end\{itemize\}" juntamente com o comando "\textbackslash item":

\begin{itemize}
\item item0
\item item1
\item item2
\end{itemize}

Gerado com o código:
\begin{verbatim}
\begin{itemize}
\item item0
\item item1
\item item2
\end{itemize}
\end{verbatim}

Para criação de listas ordenadas utiliza-se o comando "\textbackslash begin\{enumerate\} ... \textbackslash end\{enumerate\}" juntamente com o comando "\textbackslash item":

\begin{enumerate}
\item item0
\item item1
\item item2
\end{enumerate}

Gerado com o código:
\begin{verbatim}
\begin{enumerate}
\item item0
\item item1
\item item2
\end{enumerate}
\end{verbatim}

Podem criar-se tabelas aninhadas que, caso numeradas, criam numeração automatica das tabelas internas com um simbolo diferente:

\begin{enumerate}
\item item0
\begin{enumerate}
\item item00
\item item01
\begin{enumerate}
\item item010
\item item111
\item item212
\end{enumerate}
\item item02
\begin{itemize}
\item item020
\item item121
\item item222
\end{itemize}
\end{enumerate}
\item item2
\end{enumerate}

Gerado com o código:
\begin{verbatim}
\begin{enumerate}
\item item0
\begin{enumerate}
\item item00
\item item01
\begin{enumerate}
\item item010
\item item111
\item item212
\end{enumerate}
\item item02
\begin{itemize}
\item item020
\item item121
\item item222
\end{itemize}
\end{enumerate}
\item item2
\end{enumerate}
\end{verbatim}

\subsection{Tabelas}
As tabelas em \LaTeX são um comando ligeiramente mais complexo que as listas. Uma tabela simples começa com o comando "\textbackslash begin\{table\}[mod] ... \textbackslash end\{table\}". Este modificador geralmente é "h" para "here", indicando que a tabela vai ser posta, se possivel, no lugar relativo ao codigo que a descreve. No interior deste comando é inserido outro "\textbackslash begin\{tabular\}\{tipo\}\textbackslash end\{tabular\}", seja o tipo a forma como os items estão dispostos horizontalmente, com ou sem linha vertical a separa-los, alinhados ao centro (c), à esquerda (l) ou à diretia (r). Para além destes é ainda possivel decidir quando existe uma quebra de linhas com o comando "\textbackslash \textbackslash" e se existirá uma linha horizontal entre as duas linhas com o comando "\textbackslash hline", items são separados por um e comercial "\& ":

\begin{table}[h]
\center
\begin{tabular}{|c|l|r}
\hline
  & 0 & 1 \\ \hline
0 & 00 & 10 \\ \hline
1 & 01 & 11 \\ 
\end{tabular}
\end{table}

Gerado com o código:
\begin{verbatim}
\begin{table}[h]
\center
\begin{tabular}{|c|l|r}
\hline
  & 0 & 1 \\ \hline
0 & 00 & 10 \\ \hline
1 & 01 & 11 \\ 
\end{tabular}
\end{table}
\end{verbatim}




