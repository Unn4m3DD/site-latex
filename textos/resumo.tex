Este relatório destina-se a explorar as funcionalidades da linguagem \LaTeX.
Desde indexação de texto, inclusão de referências num ficheiro externo, criação de tabelas, introdução de símbolos matemáticos e até mesmo criar gráficos com pontos vetores e arrays destes. É um documento que foca na utilização prática da tecnologia e que visa introduzir pessoas novas à mesma. Apesar de já existirem manuais de extrema qualidade como \cite{tnssl} ou \cite{semautor}, geralmente pecam por se tornarem listas quase exaustivas de comandos enquanto o objetivo deste documento é apresentar a linguagem e dar sujestões de utilização como na  \autoref{chap.projetoscomlatexcomandoinput} em que é explorada a potencialidade do comando "\textbackslash input".

Este documento não se propõe de forma alguma a substituir o manual de \LaTeX  \cite{Lamport.94} , é apenas um guião simples para iniciantes na linguagem e na forma de escrever textos, uma vez que a grande maioria conhece e utiliza ferramentas como Microsoft Word que utiliza o paradigma \ac{wysiwym}.

O documento tem algum ênfase nas funções de matemática por se tratarem de funções que produzem texto mais elaborado e com mais variantes que podem confundir utilizadores que acabam de ingressar na linguagem. Para além desta a maioria das funções mais gerais de \LaTeX serão abordadas ainda que com menos detalhe.

\footnote{Este relatório serve também como exercício prático dos conhecimentos adquiridos para a sua realização.}
