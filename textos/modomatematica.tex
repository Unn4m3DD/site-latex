

O modo de matemática permite escrever símbolos e fórmulas de matemática de forma explícita e não ambígua que após compilados se assemelham extremamente a forma escrita dos mesmos

\subsection{Como utilizar o Modo de Matemática}
Existem dois modos de escrever matemática em \LaTeX , um é o \ac{ilmm} que é iniciado terminado por um único cifrão, "\$", e que vai escrever a fórmula na linha onde é escrita reduzindo a formatação de texto para que encaixe na linha, o outro é o \ac{nlmm} que inicía e termina com um cifrão duplo "\$\$" e vai criar a fórmula centrada numa nova linha com toda a formatação possível. Pode ainda usar-se o comando "\textbackslash begin\{equation\} ... \textbackslash end\{equation\}", este comando será discutido na \autoref{subchap.indexmat}.

Por exemplo no \ac{ilmm} "\textbackslash int\textbackslash limits\_\{a\}\textasciicircum\{b\} f(x)" apareceria da seguinte forma: "$\int \limits_{a}^{b} f(x)$"\\ enquanto que no \ac{nlmm} seria:
$$\int \limits_{a}^{b} f(x)$$

Existe ainda uma forma de utilizar o \ac{ilmm} com a formatação completa, o comando "\textbackslash displaystyle\{\}", o que por vezes pode criar grandes espaçamentos entre linhas.
Por exemplo o comando "\textbackslash displaystyle\{\textbackslash int\textbackslash\,limits\_\{a\}\textasciicircum\{b\}\,f(x)\,\}" \ \mbox{tornar-se-ia} "$\displaystyle{\int \limits_{a}^{b} f(x)}$"

\subsection{Comandos mais comuns em \LaTeX}
Alguns dos comandos mais comuns do modo de matemática do \LaTeX são

\begin{table}[ht]
\centering
\begin{tabular}{| c | c | c | }
    \hline
    Código                   & \ac{ilmm}          & displaystyle \ac{ilmm}         \\ \hline
    \textbackslash sqrt\{x\} & $\sqrt{x}$         & $\displaystyle{\sqrt{x}}$     \\ \hline
    x\textasciicircum p      & $x^{p}$            & $\displaystyle{x^{p}}$      \\ \hline
    x\_ b                       & $x_{b}$            & $\displaystyle{x_b}$           \\ \hline
    x\textasciicircum p\_ b  & $x_b^{p}$        & $\displaystyle{x_b^{p}}$      \\ \hline
    \textbackslash pi          & $\pi$             & $\displaystyle{\pi}$          \\ \hline
    \textbackslash otimes    & $\otimes$         & $\displaystyle{\otimes}$     \\ \hline
    \textbackslash cup       & $\cap$             & $\displaystyle{\cap}$          \\ \hline
    \textbackslash subset    & $\subset$         & $\displaystyle{\subset}$     \\ \hline
    \textbackslash sum       & $\sum$             & $\displaystyle{\sum}$          \\ \hline
    \hline
\end{tabular}
\end{table}

\subsection{Caracteres do Modo Matemática}
Em matemática regularmente é necessária a introdução de alguns caracteres diferentes dos \ac{ascii} que temos no teclado. Geralmente estes são letras gregas que podem ser obtidas pelo seu nome em inglês, no geral, pelo comando "\textbackslash nome" para letras minúsculas e pelo comando "\textbackslash Nome" para letras maiúsculas.

Exemplificando:
\begin{table}[h]
\center
\begin{tabular}{|c|c|}
\hline
Código & Output \\ \hline
\textbackslash phi    & $\phi$ \\ \hline
\textbackslash Phi    & $\Phi$ \\ \hline
\textbackslash delta & $\delta$ \\ \hline
\textbackslash Delta & $\Delta$ \\ \hline
\end{tabular}
\end{table}

Podem também introduzir-se caracteres como infinito e "tender para" com os comandos "\textbackslash infty" ($\infty$) e "\textbackslash to" ($\to$).

Em matemática uma forma comum de representar somatórios é com uma soma incluindo 3 pontinhos no meio da fórmula:
$$\sum_{i = 0}^{n} i = 1 + 2 + 3 + \dots + n$$
Estes 3 pontinhos podem ser obtidos com o comando "\textbackslash dots". Variantes deste símbolo com pontos na vertical e diagonal para matrizes por exemplo conseguem-se com as variações "\textbackslash vdots" \, $\vdots$ \,, "\textbackslash ddtos" \,$\ddots$\, e "\textbackslash reflectbox\{\$\textbackslash ddots\$\}"\, \reflectbox{$\ddots$}\\

Podem representar-se desigualdades como o comando "\textbackslash tipo"\, seja o tipo as iniciais da desigualdade em inglês:
\begin{table}[h]
\center
\begin{tabular}{|c|c|c|}

\hline
Código & Output & Nome Em Inglês \\ \hline
\textbackslash ne & $\ne$ & Not Equal \\ \hline
\textbackslash leq & $\leq$ & Lesser or EQual \\ \hline
\textbackslash geq & $\geq$ & Greater or EQual \\ \hline
\textbackslash equiv & $\equiv$ & EQUIValent \\ \hline

\end{tabular}
\end{table}

Outros caracteres que podem ser necessários entram na categoria de caracteres compostos, por exemplo o $\pi$ maiúsculo, normalmente usado na matemática como produtório. Este tipo de caracteres será discutido na \autoref{subchap.integrais}.

\subsection{Frações}
Para introduzir frações é utilizado o comando "\textbackslash frac\{numerador\}{denominador}"\\

$$\frac{numerador}{denominador}$$

Em     \LaTeX podem fazer-se frações dentro de frações, aninhando (do inglês nest) comandos:

$$\frac{a + \frac{b}{c}}{d}$$

Com este tipo de comando deve ter-se especial cuidado com o \ac{ilmm} pois pode tornar-se ilegível, por exemplo em: "$\frac{\frac{\frac{a}{b}}{c}}{\frac{d}{\frac{e}{f}}}$"  \,, neste caso é aconselhada a utilização do "\textbackslash displaystyle\{\}" para cada fração: "$\displaystyle{\frac{\displaystyle{\frac{\displaystyle{\frac{a}{b}}}
{c}}}{\displaystyle{\frac{d}{\displaystyle{\frac{e}{f}}}}}}$"\,ou o \ac{nlmm}. De notar que o comando com frações e displaystyles alinhados é extremamente confuso:
\begin{verbatim}$\displaystyle{\frac{\displaystyle{\frac{\displaystyle{\frac{a}{b}
}}{c}}}{\displaystyle{\frac{d}{\displaystyle{\frac{e}{f}}}}}}$.
\end{verbatim}

\subsection{SuperScripts e SubScripts}
No modo de matemática do \LaTeX podem ser usados caracteres escritos como expoente ou base (superscript e subscript) utilizando os caracteres \textasciicircum \, e \_ respetivamente.
Podem fazer-se combinações de ambos e incluir vários caracteres como expoente ou base. Um exemplo de utilização composta de tudo isto seria:
$$log_e(e^{Exemplo Composto})$$
Gerado com o código:
\begin{verbatim}$$log_e(e^{Exemplo Composto})$$
\end{verbatim}

\subsection{Radicais}
Em \LaTeX radicais são criados com a função "\textbackslash sqrt[índice]\{x\}".
Por exemplo: $$ \sqrt[3i]{x^2+1}$$
Gerado com o código:
\begin{verbatim}$$ \sqrt[3i]{x^2+1}$$
\end{verbatim}

\subsection{Integrais, Somatórios, Produtórios e Limites}
\label{subchap.integrais}
O uso de integrais e somatórios em \LaTeX tem as suas particularidade e pode variar com de acordo com gosto pessoal. A forma mais simples de representar um integral, um somatório, um produtório e um limite é com os comandos "\textbackslash int"\,($\int$) , "\textbackslash sum"\,($\sum$), "\textbackslash prod"\,($\prod$) e "\textbackslash lim"\,($\lim$).

Para além destas existem também as formas compostas com limites superiores e inferiores que podem ser representadas de duas formas, utilizando os já vistos superscript e subscript ou um novo comando chamado "\textbackslash limits \textasciicircum \_ ". O comando "\textbackslash displaystyle" também afeta a forma como é apresentada a fórmula. A única diferença entre os dois comandos é a forma de representação final.\\

Ao utilizar os comandos de superscript e subscript temos:
$${\int^b_a f(x) dx = \lim_{||\Delta x|| \to 0} \sum_{i=1}^{n} f(x_i^*)\Delta x_i}$$
Gerado com o código:
\begin{verbatim}$${\int^b_a f(x)dx=\lim_{||\Delta x||\to 0}\sum_{i=1}^{n}f(x_i^*)
\Delta x_i}$$
\end{verbatim}

Com os comandos de limite temos:
$${\int \limits^b_a f(x) dx = \lim \limits_{||\Delta x|| \to 0} \sum \limits_{i=1}^{n} f(x_i^*)\Delta x_i}$$
Gerado com o código:
\begin{verbatim}$${\int \limits^b_a f(x)dx=\lim \limits_{||\Delta x||\to 0}\sum
\limits_{i=1}^{n}f(x_i^*)\Delta x_i}$$
\end{verbatim}

\subsection{Indexação de Equações}
\label{subchap.indexmat}
O modo de matemática do \LaTeX , munido da biblioteca \AmS -\LaTeX , oferece também a possibilidade de referência a equações através do comando "\textbackslash eqref\{ref\}". Para isso é necessário criar uma secção de texto com o comando "\textbackslash begin\{equation\} ... \textbackslash end\{equation\}":

\begin{equation}
    \sum_{i=1}^\infty \label{somatório}
\end{equation}

Gerado com o código:
\begin{verbatim}
\begin{equation}
    \sum_{i=1}^\infty \label{somatório}
\end{equation}
\end{verbatim}

Pode referenciar-se a equação pelo seu label com o comando
"\textbackslash eqref\{ somatório\}\,", aparecendo no documento final desta forma  \eqref{somatório}

Para mudar a forma como a equação é numerada é possível utilizar o comando "\textbackslash tag{tag}" que substitui a numeração pelo conteúdo da tag como no exemplo:
\begin{equation}
    \sum_{i=1}^\infty \label{somatorio2} \tag{somatorio}
\end{equation}

Gerado com o código:
\begin{verbatim}
\begin{equation}
    \sum_{i=1}^\infty \label{somatorio2} \tag{somatório}
\end{equation}
\end{verbatim}

Será referenciado na mesma pelo comando "\textbackslash eqref\{ somatorio2\}\," desta forma \eqref{somatorio2}.
É necessário ainda o cuidado com a ordem do label e da tag que têm de ser explicitamente label $\to$ tag.

\subsection{Matrizes}
Para representar matrizes em \LaTeX utiliza-se o comando "\textbackslash [ \textbackslash begin\{bmatrix\} ... \textbackslash end\{bmatrix\} \textbackslash ]" e uma estrutura bastante semelhante às listas:

\[
\begin{bmatrix}
a & b & c \\
d & \ddots & \vdots \\
e & \dots & f \\
\end{bmatrix}
\]

Gerado com o código:
\begin{verbatim}
\[
\begin{bmatrix}
a & b & c \\
d & \ddots & \vdots \\
e & \dots & f \\
\end{bmatrix}
\]
\end{verbatim}

\subsection{Chavetas}
Em \LaTeX existem vários tipos de chavetas, a mais simples é a chaveta utilizada para representar sistemas de equações com o comando "\textbackslash begin\{cases\} ... \textbackslash end\{cases\}":


$$
f(x) = \begin{cases}
1 & \text{ , se } x < 2 \\
b & \text{ , se } x \le 2 \\
\end{cases}
$$
\\
Gerado com o código:
\begin{verbatim}
$
f(x) = \begin{cases}
1 & \text{ , se } x < 2 \\
b & \text{ , se } x \le 2 \\
\end{cases}
$
\end{verbatim}

Podem ainda usar-se chavetas em cima e em baixo de equações com os comandos "\textbackslash underbrace" e "\textbackslash overbrace":

$$ \overbrace{a+a}^{2a} + \underbrace{b+b}_{2b} = 2a + 2b$$

Gerado com o código:
\begin{verbatim}
$$ \overbrace{a+a}^{2a} + \underbrace{b+b}_{2b} = 2a + 2b$$
\end{verbatim}







