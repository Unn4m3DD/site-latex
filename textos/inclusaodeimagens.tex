
Imagens são elementos essenciais em quase todos os documentos científicos. O \LaTeX providência várias opções para manipular imagens e faze-las parecer exatamente como o utilizador precisa,no entanto,o \LaTeX não possui nenhum mecanismo nativo de inclusão de imagens, temos que recorrer ao uso de packages.\cite{labi} Neste subcapitulo vamos explicar como incluir imagens nos formatos mais comuns , como encolhe-las ,como as aumentar ,como as rodar, e ainda como as referenciar nos documentos.

Vamos usar como exemplo o uso do logotipo da UA.

\begin{verbatim}
Aqui - \includegraphics[scale=0.5]{ua.pdf}
 - o logotipo da UA.
\end{verbatim}

que origina o resultado :

Aqui - \includegraphics[scale=0.5]{ua.pdf} - o logotipo da UA.

Também é possivel utilizar esta imagem dentro de de um objeto flutuante:

\begin{verbatim}
\begin{figure}[h]
\center %imagem centrada
\includegraphics[scale=0.5]{ua.pdf}
\caption{Logotipo da UA} %legenda
\label{fig:ualogo.1}
\end{figure}
\end{verbatim}

\begin{figure}[h]
\center %imagem centrada
\includegraphics[scale=0.5]{ua.pdf}
\caption{Logotipo da UA} %legenda
\label{fig:ualogo.1}
\end{figure}

Na inclusão de um ficheiro é usual indicar o seu nome sem extensão (nos casos acima o nome
completo do ficheiro é ua.pdf e está na pasta do próprio documento). Com efeito, podem existir
vários ficheiros para a mesma imagem, cada um com o seu formato, e deste modo facilita-se a
compilação do documento para formatos de saída diferentes.
O comando de inclusão de imagens possui várias opções, entre as quais as que permitem redimensionar
ou de outra forma ajustar a imagem a incluir:

\textbf{height} - altura da imagem.

\textbf{width} - largura da imagem.

\textbf{scale} - fator de escala.

\textbf{angle} - ângulo de rotação.


Mais a baixo , na figura 3.2 vamos mostrar vários exemplos uteis de como a inclusão de imagem funcionam:

\begin{verbatim}
\begin{figure}[h]
\center % Centra as imagens
a) \includegraphics{ua.pdf}
b) \includegraphics[height=2cm]{ua.pdf}
c) \includegraphics[width=10mm]{ua.pdf}
d) \includegraphics[scale=.5,angle=90]{ua.pdf}
e) \includegraphics[height=5mm,width=2cm]{ua.pdf}
\caption{Logotipo da Universidade de Aveiro: a) na dimensão real,
b) com 2cm de altura, c) com 10mm de largura, d) com altura e largura
reduzidas a $1/2$ e simultaneamente rodado 90º e e) com uma modificação
anamórfica da altura e da largura.}
\label{fig:ualogo.2}
\end{figure}
\end{verbatim}

\begin{figure}[h]
\center % Centra as imagens
a) \includegraphics{ua.pdf}
b) \includegraphics[height=2cm]{ua.pdf}
c) \includegraphics[width=10mm]{ua.pdf}
d) \includegraphics[scale=.5,angle=90]{ua.pdf}
e) \includegraphics[height=5mm,width=2cm]{ua.pdf}
\caption{Logotipo da Universidade de Aveiro: a) na dimensão real,
b) com 2cm de altura, c) com 10mm de largura, d) com altura e largura
reduzidas a $1/2$ e simultaneamente rodado 90º e e) com uma modificação
anamórfica da altura e da largura.}
\label{fig:ualogo.2}
\end{figure}



