Devido ao seu cariz principalmente académico, \LaTeX suporta a inclusão de referências a bibliografia, utilizando o comando de referência "\textbackslash cite\{nome\}" em conjunto com o comando de definição "\@ misc\{nome, atributos\}", como por exemplo:

\begin{verbatim}
%comando de definição
@misc{tnssl,
    author={{Tobias Oetiker}},
    title={{The Not So Short Introduction to \LaTeX}},
    month={Março},
    year={2018},
    note = "Online; acedido em Novembro de 2018]"
}

%comando de citação
\cite{tnssl}
\end{verbatim}
 
O resultado obtido com a citação é semelhante a este \cite{tnssl}

Podemos incluir as definições num documento externo através do comando "\textbackslash input\{\}" (aprofundado na \autoref{chap.projetoscomlatexcomandoinput}) mas o mais comum é criar um ficheiro com a extensão ".bib" e utilizar o comando "\textbackslash bibliography\{filename\}" para referenciar que é um ficheiro de bibliografia.

Utiliza-se o comando "\textbackslash printbibliography" para indicar ao compilador onde deve ser escrita a bibliográfica caso referenciada.

