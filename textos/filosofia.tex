Como qualquer editor de texto \LaTeX tem as suas vantagens e desvantagens em relação a outros editores , portanto neste subcapítulo vamos enumerar algumas das suas vantagens e desvantagens e vamos comparar um pouco com o Word.

\subsection{Prós}
\textbf{Velocidade e automatização:} Um arquivo tex é rápido, já que trata de texto apenas. O output, em pdf, é obviamente mais rápido do que Word. Links entre secções (também possíveis em Word) são facilmente implementados. O mesmo pode ser dito sobre figuras e gráficos, por exemplo.

\textbf{Layout e ferramentas científicas:} \LaTeX usa um layout mais profissional. A interação entre diferentes tipos de objeto e o texto flui de forma excelente, algo raro em Word.

\textbf{Compatibilidade:} Diferentes versões de Word podem causar mudanças bruscas de formatação em doc. TeX, por outro lado, não sofre com esse tipo de problema. OK, pacotes são atualizados, e alguns detalhes podem mudar, mas o nível de compatibilidade é suficientemente baixo para que se possa simplesmente presumir que seja zero, principalmente quando comparamos com docs. Um tex pode ser aberto em virtualmente qualquer lugar. Pode-se editar mesmo sem ter LaTeX instalado, porque não é preciso compilar um tex para o editar: o resultado final (pdf) é um arquivo independente do arquivo com o conteúdo (tex).

\textbf{Pacotes:} LaTeX é gratuito (open source). Há milhares de pacotes disponíveis para as mais variadas tarefas. Diferentes desenvolvedores podem adicionar funcionalidades a partir de pacotes,essa é a grande vantagem de sistemas open source (há outras, é claro). O número de desenvolvedores/programadores no trabalhar no Word é limitado, o que significa que bugs demoram mais a serem corrigidos. Em sistemas abertos, como a Wikipedia, há um número absurdo de pessoas a trabalhar constantemente para que a coisa toda funcione.

\textbf{Aprendizagem:} Usar LaTeX é aprender, constantemente, coisas diferentes. Se o utilizador nunca programou, utilizar LaTeX será uma introdução básica: a partir daí possivelmente poderá partir para outras linguagens depois. Isso porque o feedback é bastante instantâneo: aprende-se algo, compila-se , e percebe-se que conseguiu criar uma estrutura bastante complexa. Isso é estimulante, como qualquer atividade em que se aprende constantemente. Além disso, para pessoas que não são da área  computação/programação, usar LaTeX é uma ótima oportunidade de "pensar" em códigos.Passa-se a entender uma nova sintaxe, e aplicar seus conhecimentos intuitivamente para criar aquilo que o Word não consegue. Isso é excelente não apenas para artigos/teses/dissertações, mas para handouts e apresentações de slides: muito do que usamos em Linguística exige uma certa complexidade gráfica, e transmitir isso nem sempre é intuitivo, principalmente em editores de texto como o Word.

\subsection{Contras}
Basicamente, a desvantagem de \LaTeX é a interface. Se o utilizador tem pouco conhecimento na área e sabe apenas o básico, dificilmente irá gostar de usar \LaTeX. É preciso paciência e dedicação no início, coisas que antes eram simples são, de repente, complicadas. Isso é comum a qualquer linguagem de programação.

Por outro lado, se o utilizador gosta de desenvolver diferentes habilidades e se for avançado na área, \LaTeX é uma ótima opção. Este tipo de recurso está normalmente associado a estudantes que pretendam fazer progressão de estudos nomeadamente Mestrados e Doutoramentos. Nesse caso, certamente têm a capacidade para aprender uma linguagem como \LaTeX. Ou seja, as pessoas que normalmente procuram \LaTeX têm o nível de instrução necessário para a aprender, pois, as pessoas que mal sabem usar um computador não devem estar a pensar em publicar artigos científicos.

Se os utilizadores lidarem com pessoas que não utilizam \LaTeX ,nomeadamente em trabalhos de grupo, isso pode ser um problema. A solução é utilizar pdfs para comentários… o que talvez não seja ideal, dependendo do tipo de elementos que certo grupo tenha.

\subsection{WYSIWYM}
Esta expressão \ac{wysiwym} em inglês significa, basicamente, que a exibição na tela deve primar pelas informações, e não pela formatação, que é um trabalho que deve ser deixado para o computador.

O que quer dizer que o utilizador escreve o conteúdo numa forma estruturada, marcando-o de acordo com o seu significado e a sua importância no documento, deixando a sua aparência final desde um ou mais estilos de folhas em separado.

A maior vantagem deste sistema é a separação total entre apresentação e conteúdo: os utilizadores podem estruturar e escrever o documento uma vez ,em vez de constantemente alterá-lo cada modo de apresentação.

\subsection{WYSIWYG}
Num documento  \ac{wysiwyg} implica que uma interface do utilizador permite que o utilizador veja o produto final muito semelhante enquanto o documento está a ser criado.No geral,isto implica a habilidade direta de manipular um layout de um documento sem ter de escrever ou lembrar de nomes de comandos de layout.O significado real depende da perspetiva do utilizador.

