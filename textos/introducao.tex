\section{História do \LaTeX}
A linguagem de descrição de texto \LaTeX começou a ser desenvolvida com o nome de \TeX  no ano de 1977 e publicada em 1982 por Donald E. Knuth. \LaTeX é uma linguagem que utiliza o código presente na linguagem \TeX criando macros que facilitam a escrita de documentos. \TeX foi criada com o intuito de tornar artigos de jornais e revistas mais atraentes ao público e eventualmente, com o surgimento do \LaTeX tornou-se uma ferramenta utilizada mundialmente para realização de textos académicos devido aos seus inúmeros recursos como indexação automática de texto, figuras, criação automática de índices, entre outros. Ainda a ressaltar duas curiosidades sobre a linguagem \TeX , uma é a forma como é descrita a versão da mesma, a versão tende para $\pi$ e está atualmente na versão 3.14159265. A outra é a forma como se pronuncia a palavra, a letra X é lida com som de "c" já que a letra grega chi "$\chi$" tem esta pronúncia.

\section{Motivação e Descrição Capítulos}
O trabalho foi realizado com objetivo de aprofundar o conhecimento da linguagem \LaTeX criando também um documento que facilite outros a conhecer melhor a linguagem.

Na \autoref{chap.filosofia} está descrita a filosofia por detrás da linguagem como o facto de ser uma linguagem open source e uma markup language que implementa o conceito \ac{wysiwym}.

Na \autoref{chap.estruturadecomandos} é apresentada a estrutura fundamental dos comandos \LaTeX que são fundamentais para a compreensão do funcionamento da linguagem.

Na \autoref{chap.pacoteslatex} está descrita a forma como é possível reaproveitar o código de terceiros para introduzir imagens, fórmulas matemáticas mais complexas, criar gráficos, entre outros.

Na \autoref{chap.caracteresespeciais} será mostrado como introduzir caracteres especiais desde acentos (pouco comuns no inglês) até texto  $^{superscript}$ recorrendo ao modo de matemática.

Na \autoref{chap.inclusaodeimagens} é feita uma breve abordagem à inclusão de imagens externas no documento.

Na \autoref{chap.listasetabelas} são descritas as tabelas em detalhe desde posicionamento do texto na tabela até à forma como estão dispostas as divisórias.

Na \autoref{chap.modomatematica} é apresentado em algum detalhe o modo de matemática que permite escrever fórmulas e símbolos matemáticos dos mais variados tipos.

Na \autoref{chap.criacaograficos} há uma breve abordagem à criação de gráficos em \LaTeX mostrando o conceito e alguns comandos simples.

Na \autoref{chap.projetoscomlatexcomandoinput} é apresentada uma forma de estruturar documentos \LaTeX em vários módulos mais pequenos.

Na \autoref{chap.referenciasadocumentosexternos} é descrita a forma como podem ser apresentadas bibliografias e referências a outros documentos.

Na \autoref{chap.fonteselinguagens} é apresentada a forma como se podem utilizar estilos e tamanhos de texto.

